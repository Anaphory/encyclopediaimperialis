\section{Jacks and Jills}
\subsection{Jack-of-Irons}
Jack-of-Irons was a legendary giant, also called variously Jack-in-chains and Bloody Jack. Wrapped in heavy chains and wielding a terrible spiked club, the massive creature is said to be twice the size of a person and impossible to kill. According to stories the creature recovers almost immediately from even mortal wounds, and limbs hacked off continue fighting. Popular tales suggest that the demonic horror was defeated by a girl variously known as Clever Jill and Wise Mary who tricks the beast into falling down a well from which it is unable to escape. The legend insists that the giant is still buried under the earth somewhere near the modern town of Wayford.

Interestingly, some scholars believe the creature spoken of in Suaq and Kallavesi legends that they call the skinned king may be the same creature, although in the stories of Wintermark the giant beast has no skin and is ultimately defeated by a clever hunter who creates a human-shaped mannikin of fresh deer meat studded with spiked hooks, and then taunts the giant into grappling the mannikin. Trapped by the hooks, the giant cannot escape and is rolled down a steep incline into the waters of Westmere where it is assumed to have drowned. 
\subsection{Jack and the Giant}
A long time ago, back in the days before the Empire, a great gory giant came thundering down from the mountains. Her vasty swag-belly had turned to slack skin, for winter had been hard and the hill-sheep seemed to get more nimble with each passing year. Down through the hills she came, cursing and groaning. Down through the fens she came, creeping and weeping. Down to the Marches she came, down to find the flesh of children to fill her empty belly.

But as she crossed the bawns of the Marches, who should she come upon but a young beater by the name of Jack, who was sitting down to eat his lunch upon a tree he had just felled with his axe. “Tell me, oh son of the sod, which way to the nearest village?” She rumbled (for giants have the gift of speech, even if they use it but seldom), “Quickly now, lest I crack your bones and milk their marrow for my gruel!”

Jack set his head to the side, just so, and looked up at her as if she was a tree for the felling. He saw her terrible tearing talons, like a brace of skinning knives on the end of her bony fingers. He saw her fearsome fangs, sharp and black like the beaks of a nest of crows. And he saw her great lantern eyes, rheumy and milk-stained with age.

“Ma’am, I’d be most happy to oblige you, but a favour for a favour seems fair – won’t you please help me move this log first? Then I’d be most obliged and tell you the directions you want.”

“Why should I do aught for you? I could pop off your head as easy as spitting, see if I don’t!”

“True, but then you wouldn’t have you directions, and it’s many a mile to the nearest village.”

Her stomach growled so loudly that a nearby owl fell stunned from her perch and Jack had to clap his hands to his ears to keep from being deafened. At last she sighed, “Where is this log? Be quick about it, woodsman.” Full well she intended to kill him for his impertinence, even though his bones were too tough for her old teeth to grind.

“Why, just in front of you, ma’am, if you’ll but reach for it.”

Blindly she groped her great hands across the ground.

“Where? I feel nothing.”

She reached down further, sniffing at the ground like a dog.

“You’ll just need to reach a little lower, a little lower…” and so saying, Jack grabbed his axe and struck her wicked head from her shoulders with a single blow. The thunder of her fall was so great it knocked down trees for a mile around and blew Jack clean out of his boots.

And that is the story of how Jack the beater slew the giant. He was rewarded with an axe of gold, which was not much use for cutting wood, but he wore it nonetheless, as a reminder that “There’s ne’er a tree so big it can’t be felled.” He carries it still, just ask him if you don’t believe me… 

\section{Good Walder}
\subsection{How the Oakbeck got its name}
When Good Walder travelled the Marches, Howby was a single farm South of the Southridge, just inside the Stockmarch from Hahnmark.
While the soil was good, water was scarce and the next brook was far away, and even though the people worked hard, the village was very poor.

Now Good Walder had walked all summer day with his heavy oaken club when he came to the place, and he was tired and thirsty.
When he had just sat foot inside the fence surrounding the homestead, a farmhand from the nearby barn called out to him and said, “Good traveller, why do you stand in the sun? Go thither, and sit in the shade! I will fetch you some water from the brook!”
Off she was, and so he sat down on a bench by the yeoman's house in the shade.
He had barely sat down when a house helper called out to him and said “Good traveller, why do you sit outside in the heat? Come inside, and sit on the cool clay floor! I will fetch you some water from the brook!”, and off he went with a bucket to the brook.
So Good Walder entered the house, and sat in a cool corner by the door.
But he had not sat there for a long time when the yeoman of the house saw him sitting there, and she said “Good traveller, you have found my house well and no-one has given you to drink yet? Sit on the high chairs by the cool stone wall then, and let me go fetch you a drink.”
And before Good Walder could say a thing, she was out of the door and off to the brook, as well.

Then they returned and gave him to drink, and used the remaining water for their herbs, and Good Walder saw that the people of Howby were diligent folk, virtuous and good of heart. He took them to the hillside behind the barn and, lifting his heavy oaken club over his head, drove it into the ground with so much force that the ground shook and split.
Water came up of a new spring and flowed down the Ghyll Good Walder had driven into the ground with his oaken club to the old brook, and so the Ghyll was called the Oakghyll, and the brook became known as the Oackbeck, and Howby grew and became a prosperous town.
And Good Walder went on to travel.
\section{Bolstering Bill}
Bolstering Bill is an everyman character generally believed to have been a Marcher – if she existed at all – who appears in a number of comic songs, stories and poems in various parts of the Empire. She (or very occasionally, he) is presented as a simple yet well-meaning yeoman, often a soldier, usually wielding a pole-arm, whose companions encounter any number of amusing or hair-raising scrapes and who survive or triumph with the aid of their companion, Bill combines good humour, honest camaraderie and more than a little tough love with an absolute commitment to never leaving a friend behind. While the truth behind these tales is contested, the character of Bolstering Bill is often considered to be an Exemplar of Loyalty. Even the critics of the Bolstering Bill stories note that it is likely that many of them recount the adventures of real people, and have simply been ascribed to Bolstering Bill by later generations of storytellers to make them more accessible to their audiences.

Be this how it may, Bolstering Bill was recognised as an exemplar of Loyalty by the Highborn assembly of the Virtuous before the foundation of the Empire. She has left a legacy in the form of the weapon that carries her name.

%Once there lived a widow with seven little children. She was forced to work for them in order to keep food on the table and would only have time to sew for them at night when she was home. She was so tired then that she could not spend as much time as she would have liked to. The youngest of the seven children was a little boy, Jack by name. He was a good child and never fretted in the least. But when the eldest had a new shirt made and his was passed on down the line to the next one and by the time Jack got a shirt it would almost be too worn to stay on his frail little body. Nevertheless he was a happy child and never did anything to cause his mother worry. In turn he was trying to do something for somebody else most of the time in his humble and childish way. At nights when the mother had the chance she would spin wool and make the shirt for them, and each had only one at a time. When Jack saw a sheep he would feed it a bunch of grass he pulled. When he found a bird on the ground he would place it back into its nest. He would not kill a spider but stand and wonder at the skill with which it wove its web. One day after his shirt had become so ragged it had fallen from his body he was walking by a berry bush. The weather was warm and his mother had decided to let him roam around without anything on as she had not a shirt for him to wear. He heard a voice and looked around to find a sheep by him, the sheep asked him where his shirt was and Jack told him the circumstances as best he could in his childish manner. The sheep told him he would give him some wool and pulled off his coat of wool and gave it to him and Jack went toward home with the wool in his arms. As he passed a thorn bush the child heard a voice asking him where he was going and Jack told it that he was going home to see about his shirt. The thorn told him to hand it to it and the thorn drew it thru the briars to card the wool and soon it was carded in a beautiful way. Jack went on to see what must be done to finish his shirt. He saw nearby a spider web and the old spider in the center of the web. The spider to Jack to give the carded wool to it and it would spin it into thread, he did and soon Jack had enough of the beautiful thread to make a shirt. On he went toward home and as he went near a brook he almost stopped upon a crab. The crab asked where he was going and when Jack told him he said for him to give him the cloth and he did and the crab cut the cloth into a shirt pattern with his big pincers. The Jack was very sad for he had everything but to sew the shirt together and he too knew that his mother would be so tired and busy at night that she would not be able to sew it for him. But just then a bird in a bush close-by saw him and asked where he was going and Jack told it his story. The bird took the pattern and a straw in its bill and flew back and forth until the shirt was sewed together and there his shirt was. He had been paid for his kindness to the creatures and he had one of the finest shirts that could be had by any little fellow.
\subsection{The Father's Scythe}
\textit{The editor is aware of a Mummer’s Play of this story, and would be willing to pay a troupe to perform the piece for transskription.}
Once upon a time, Bolstering Bill is hosting two brothers, one a reaperman and physick, the other a a glaive soldier and labourer. The brothers are not rich, and their greatest possession is a sharp and practical scythe, with an excellent blade and a fine handle, handed down for generations from their forebearers, which one of them used in peace, and the other in war.

Now the brothers are aware they will have to part way soon, the soldier going to war and the reaper going to harvest, and while they have enough rings to buy a second tool, they cannot agree who should be the owner of the family's scythe.

In comes our Wilhelmina. Bill had an acre of land to reap, but she only had a sickle and the brothers were staying with her, so she asked them to borrow the scythe. With the excellent blade fitted to the fine handle, the reaping was done soon, but on the last swing, Bill hit the blade against a stone and jagged the blade and chipped the handle.

She was desolate, and brought the fine handle to the woodcutter and the excellent blade to the blacksmith and asked them to repair the scythe.

When Wilhelmina returns to the artisans to receive the parts, the woodcutter said “I saw that your blade was jagged and assumed you threw it away, so I asked the blacksmith for a new excellent blade. Here is your scythe back, Bill!” and the blacksmith said “I saw that your handle was chipped, so I assumed you wanted it replaced and asked the woodcarver for a new fine handle. Here is your scythe back, Bill!”

Wilhelmina was desolate, for now she knew not which of the two scythes was the father’s scythe, and she went back to the brothers to apologise.

But the brothers were happy that Bill had solved their problem, and when they parted way, each went their own way with their ancestor's fine scythe, just with a new fine handle and a new excellent blade.

\subsection{The Army of Buxford}
Once upon a time, Bolstering Bill was going with the Buxford company to fight the Jotun. Buxford was not a fertile village, and the company could not send many yeomen to fight the barbarians. When they were in enemy territory, they had to make camp in an imperial fort, of which they filled only a puny portion.

The men and women of the company were gloomy, and not very enthusiastic about the fighting, so Bill went around the camp and tried to cheer them up. 

It was a cold and damp night, so Bill went and split many logs and gave them to the soldiers, so they lit many fires to keep themselves warm.

Then Bill found she had some pungent herbs left in her satchel, and she gave them to the cooks, and the cooks decided to also be more generous with the spices in the food, so that soon the area would smell of a big army being fed inside.

Then Bill asked the company musicians to play a tune to dance to, so that the soldiers could move their legs and be warm that day. All the musicians could play was a fast march, but that did not keep the company from dancing, and dance they did!

Then Bill found people who could tell a joke or a story, and among them were someone who had a trumpet, and another one who could make the drum roll like a mad rhino beast on the run, and another one who could make a recognisable impression of the voice of every single dog in Meade, and they all stood on a strong box and gave their spiel.

So when the Jotun scouts saw the fort in the evening, and saw the camp full of fires and laughter, happy guards on each tower, smelling of armies being fed, and sounding as if a whole menagerie of Thule war beasts was in there together with a whole elite company doing late-night marching drills, they assumed it was buzzing with imperial soldiers and advised their generals not to attack, and Bill had bolstered the Marcher army once again.
\subsection{The Twelve Pigs}
\textit{The editor is mostly aware of this story in the shape of the famous Mummer’s Play of this story, and would be happy to sponsor a performance and transscribe it.}

An Abraxus Stone is inadvertently mixed in with the feed for a herd of swine on their way to market, and the penniless protagonist must scour the countryside trying to find the pig which has eaten it in order to secure his fortune. 

Bill sticks by her friend even though she ends up having to shovel through an entire village's nightsoil to find the lost ring, and sticking by her mate trying to find a list prize pig. They look everywhere, in the foulest weather and end up nearly to the point of giving up, but Bill suggests going back home to get some warming drink before heading out again. Only to find the prize pig head down eating her friend's best apples. At least the pig got a prize for the best Apple-Fed pork!
\section{Tom Drake}
Tom Drake of Redston was a steward and general from Mitwold in the first years of the Empire.
\subsection{Alderei the Fair}
Over centuries the Volodny kept their feud alive through malign curses and spiteful sabotage. Shortly before the time of the First Empress, they launched a devastating attack against the Varushkans. While the Volodny themselves refused to use steel, they gave their support to a boyar called Alderei the Fair. Alderei raised an army and sought to unite the entire nation under his rulership, with the intention of leading this new nation to conquer the rest of humankind. Alderei was not afraid to use steel, and swept down from the north crushing everything in his path. Whenever a settlement fell to his army, he gave the survivors a simple choice “join me, or die.” To join Alderei required submission to the iron fist of a tyrant who would brook no argument, and tolerate no failure.

Many joined Alderei because, whatever else he did, he drove the monsters before his armies and protected those who swore allegiance to him from the darkness. To do this however required the aid of the Volodny and they took a terrible price from Alderei’s supporters in blood and flesh, which they used to propitiate the dark powers and fuel the boyar tyrant's armies with their malign sorceries.

It seemed certain that the Varushkans would be united under the banner of Alderei; the people lacked a unified central authority capable of resisting the conquering boyar. According to legend, a gathering of wise ones took place at this time to discuss what could be done to stem the seemingly unstoppable tide washing down from the north. The gathering lasted for a full lunar month, and eventually came down to a decision between two very different yet hauntingly similar courses of action.

The first course was to compromise with Alderei and march beneath his banner, to create a great thousand-year Empire and to offer the dark powers the sacrifices they craved. The other was to seek external aid – the Navarr had brought news of a gathering to the south where a new way of life was being discussed, one of unity rather than subjugation. In the end the wise ones took a delegation to this meeting, while the remaining free boyars who refused to accept Alderei’s yoke fought desperately against the tyrannical boyar's forces.

After days of negotiation, the wise ones agreed that Varushkans would join the nascent Empire in return for aid in defeating Alderei and his Volodny conspirators. A great host formed of the armies of the other nations rode into Varushka to relieve the besieged boyars. Tom Drake of Redston led his household and the territory's Landskeepers to Varushka. With the aid of ritual magicians from Urizen and landskeepers from the Marches, the power of the Volodny was matched and broken. They fought through unfamiliar forests, alongside all those who opposed Alderei the Fair and brought Varushka into the Empire.

General Tom Drake fought Alderai himself in a heroic battle, vorpal blade cutting steel fists, and slew him in the end, even though he lost his life in the process. The broken crown became part of the livery of the Redston folk. Several of the Volodny met their final ends as their army fell apart.

The body of Alderei was never recovered and at his execution one of the captured Volodny gave a prophecy saying that one day in the future the dark king would return and take vengeance for this defeat, shattering the Empire as he had once shattered Varushka.

\subsection{The Military Council}
The Marcher steward, Tom Drake, argued passionately that the command of Marcher yeomanry must be with a yeoman. He point blank refused to compromise; for Drake it was about the principle of representation and the right for Marchers to choose who they raised up. The military commanders of the other nations broadly agreed with him, and his vision of armies raised within nations and led by generals of those nations appointed by their senators quickly gained ground.

While Tom Drake was demanding national control of the armies, Elaine de Celadorn (Senator for Astolat) was liaising with the Imperial Senate to provide the Military Council with the civilian support she felt it desperately needed.

Despite some initial resistance, Elaine was able to convince the members of the Military Council that by handing control of the logistics of the Imperial armies to the newly formed civil service, up to and including the arrangement of their meetings, they would be freed up to focus their efforts on strategy and leadership. This lead to the creation of the Herald of the Council.
\section{Joshua Benson}
The Pickham monastery
formed out of what was
the Marcher Household of
Benson, after the formalised
acceptance of The Way across the
Marches. In particular, the Virtue of
Vigilance proved especially popular
with the household thanks to the
legend of its founder, Major Joshua
Benson.

Joshua Benson and his
family were amongst the first
yeomen to rebel from Dawnish rule.
Joshua was a boy during the Rebel
March, fighting alongside other
Marcher pioneers against the
barbarians in what would later
become the Upwold region. After
having fought back the local threat,
the company containing the Benson
family cleared and settled at a ford of
the Ashbrook, at the border of the to
Miaren forest to the East.
\subsection{The Shield and Bell}
Joshua Benson is most famous
for two battles of Pickham.
When the Dawnish nobility
had mustered their strength and lead
by their king followed the Marchers
in the spring after the rebellion,
hoping to shepherd home those of
the rebellious yeomen that would
have survived the barbarian
encounters and hardships of the
winter just passed, they were met
with a surprisingly strong
resistance, fueled by loyalty and
determination of small groups of
settlers.

When the main force
was heading for the settlement
known nowadays as the King’s
Stoke, a clank of knights-errant
tried a flanking maneuver in order
to arrive at the Western edge of the
Stoke. This maneuver might have
significantly weakened the proud
defenders, if not for the Vigilance of
Joshua Benson.

The youngster was
tending pigs of the farmstead in the
forest as he
often did,
without
ever losing
any pig.
When he
noticed the clank
walking in the direction of the ford,
he ran there started a hue and cry,
banging his stick on an iron shield
for lack of a bell.

The people of
Pickham rose and donned their
armour and weapons, and stood
firm to preuent the knights from
crossing the Ashbrook, while Joshua
returned to tend the pigs. Due to the
early warning, the folks of
Pickham prevailed and the clank had
to retreat. The battle of King’s
Stoke was won by the local
Marchers, and young Joshua Benson
did his contribution without even
being there.

While the people
of the Marches in those old days did
not yet follow the Way of the Seuen
Virtues, they could nonetheless
recognise Joshua Benson as a good
and virtuous man, and made him
the Major of Pickham. This position
corresponds to what Marchers of
today would probably refer to as the
Steward, but in those days no
uniform name for such a position had
developed yet.

Following the example
set before the arrival of the Marchers
by the Sentinel through the old tower at King's Stoke, and
inspiring many other settlements of
what we call the Eastern Guard
now, the Major had a tower erected
at the ford in subsequent years.
Throughout his life, he was wise
in leading the village, but a
particular institution of his was to
put a large bell on the tower, which
would be manned by a villager on
watch always, and when raiders,
fires or other dangers threatened the
village it would be rung in
different ways, informing not only
about the presence, but also the type
of the threat.

\subsection{Battle of Pickham}
When he was already of
advanced age (and even
after he had given up the
duty of the Steward to the next generation), the
Major – as he is still usually referred
to by the people of Pickham until
this day – would still take upon
himself more than a fair share of the
watch tower duty. During his
watch one autumn night Major
Benson, still of very good eyesight
and hearing, saw a Dawnish war
band of house Arwood approach
from the East.

The knights and
yeomen had crossed the Miaren to
make a good fortune by claiming the
place or at least returning home
with valuables. Their group
outnumbered the able-bodied men
and women of Pickham by the
double, and without Joshua
Benson's vigilance and cleverness
the village would have stood no
chance. The Major sounded the shield,
awakening all the folk of Pickham.
The fighting men and women of 
The whole village rose while
the enemy was still approaching.

Pickham guarded the ford. The
halberdiers prepared to stand their 
ground, while the children
accompanied the archers into the
tower and prepared to throw rocks 
at the inuaders. A few of those
unable to fight took position at the
mill. When the Dawnish
reached the river, they tried to cross
it. First, a few were washed away 
into deeper water when the current 
suddenly became stronger. The
elderly of Pickham had opened the
millpond weir, flooding the ford.

When the more steadfast Dawnish
approached the other side of the river,
they were surprised by the
halberdiers and archers on the river
bank. After a long fight and
significant losses on both sides, the
attackers fell back to the other side of
 the Ashbrook, tending the wounded
and hoping to gather strength for a
 second attack. Joshua Benson,
remembering his experience before
and during the rebel marches, had
recognised the lord and the hopeless
desires of some his
folk-at-arms to gain
appreciation.

He
put on a piece of
Arwood liuery from a
dead washed-off
yeoman, crossed the
river and hid on the
other side. Under the
couer of darkness, he
approached one of the
men-at-arms guarding the tent of
the lord in command. Striking up a
conversation with the young
yeoman, he saw that he had judged
correctly. Edward Stocker, as was
his name, desired to gain the respect
of his lord through glory in battle.

Benson, recognising the hopelesness
this desire, suggested the following
bet to Edward: Joshua would pretend
to be one of lord Arwood’s retinue
and approach the lord. If his lord did
watch his underlings and their deeds,
fulfilling his oath as their liege, he
should be able to recognise Joshua as
an intruder, and Joshua would
arrange that it would be Edward
who would catch him. If Joshua
managed to stay unrecognised,
Edward could consider his oath of
allegiance void. He would help
capture lord Arwood, and become a
Marcher of Pickham, where his
skills would be valued.

Edward
agreed. The Major thus
approached the lord, who was busy
consuming plundered food,
unhindered. He warned him that he
had seen a single enemy scout in a
certain direction, yonder in the
woods. Lord Arwood, eager for glory,
decided to capture the target himself,
only to be followed, and
overpowered by Joshua and Edward
just outside the camp. The Stockers became Marchers of Pickham, and Dexter Stocker made a name for himself in Anvil as a competent landskeeper.

\subsection{Epilogue}
Often as told in the
Marches, the story ends
here, with a mocking
remark about the vanity of the
Dawnish. It is however
releuant to note that lord Arwood
showed ambition and courage in his
pursuit of the foray on Pickham.
Furthermore, when captured, he
showed insight. He understood that
his defeat had happened because he
was disloyal to those loyal to him,
and went forth changing his ways.
His ransom was not taken by the
people of Pickham, but split among
the poor serfs of house Arwood,
because it had been them who had
toiled for the lord's wealth, and it
had become Marcher wisdom already
that prosperity requires working for
it. To this day, the yeomen of house Arwood
sing the song of the “Hero of
Culwich”, who took from the nobles
and gave to the peasant, but this
may be conflating or Major with
another folk hero. But just as the
monks of Pickham are known for
their vigilance to this day, so are the
lords and ladies of house Arwood for
loyalty.

\section{Henry Talbot}
Henry Talbot of Mitwold fought bravely and loyally to defend his companions, even when mortally wounded. His harness was allegedly seen fighting the Druj even after his companions buried him, and his Heart kept on bleeding long after it was returned to the Marches.
\subsection{Death of a Hero}
Ned shook his head, feeling the wounds under the brigantine. “The gambeson and the brig are holding everything closed Henry. Without the full services of the hospital in Anvil I won't be able to close them fast enough if we take it off. I'm so sorry my friend.”

Henry Talbot sat with his back to the tumbled wall of the burnt out homestead, sweat plastering his dark hair to his forehead.

The flush of vigorous life had faded from his cheeks during the escape from the barbarians, but mischief still sparked defiantly in his eyes. Sun beat down and wisps of steam made thin halos about their heads. “Nothing you could have done Ned. Thank the boys for carrying me all this way... How long do you think?”

“It could be an hour, or as short as you...”

Ned paused as he reached into his pouch of herbs. A decade of hunts made him grab instinctively for his pike, the mating cry of a Bregesland Dove in the wrong season could only mean one thing.

“Cullach-A-Cullach!” Sounded to the right, accompanied by a gurgling squeal. A moment later two Orc scouts and a human slave burst from the bushes in front of them. Ned brought the pike to bear, but not fast enough.

The smaller of the two orcs, face split with a recent wound, shoved the slave into the pike and drove her body onto the shaft. Ned drew his dagger, but the orcs were on him before the sword was free. Not relishing the prospect of a standing fight they rushed forward and bore Ned to the earth.

In a flash decision Ned dropped the dagger, grasping the wrist holding a rusty knife and breaking the nose of the second Orc. Broken Nose rolled away from the melee, streaming blood and curses, while Split Face and Ned struggled over the knife.

Suddenly Henry was stood astride Split Face, Ned's dagger in hand. With a butcher's confidence he sliced Split Face's neck to the bone, drenching Ned in ichor.

Ned held fast to the rusty knife, taking it from unresisting hands. As he cleared his eyes Ned heard Henry grunt and the tip of a blade emerged from his stomach. Henry toppled and Broken Nose stooped over the to pull the notched sword free.

Roaring in fury, Ned kicked Split Face's body off and launched himself forward. The notched sword was trapped and useless between them, allowing Ned free reign with the rusty knife...

Ham came to check on them once the rest of the orc scouts were disposed of. Ned sat beside the ashen body of Henry Talbot, truely still now that he had begun his journey through the labyrinth. The remains of a human and two orcs lay in the remains of the herb garden. Not a word passed between the Cullach, just a look and a nod.

With great care they removed Henry's harness and hid it up the chimney of the homestead, suspended from the meat smoking hooks. Ham took the small antler handled knife from Ned's shaking hand and cut Henry's warm heart free from his chest. Ned knelt and filled Henry's mouth with Marcher soil, replaced in the oiled pouch by the heart. They closed his eyes and pushed the remaining wall of the homestead over him.

The fallen stones would conceal him from desecration and keep him from scavengers. Now they had to move on, their burden heavier than a dying friend.
“Rest well, Henry Talbot, your march is done.”

\chapter{Plays}
Abraxus Stone

\chapter{Songs}
One Marcher song recounts the tale of the famous smith Anna of Ashill and her decision to move to the dankest corner of Bregasland after being asked to fix a series of increasingly unlikely objects, often improvised by the singers, using her magical Redsteel Chisel. 
\end{document}
%%% Local Variables: 
%%% coding: utf-8
%%% mode: latex
%%% TeX-engine: xetex
%%% TeX-master: "encyclopedia"
%%% End: 
